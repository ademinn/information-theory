\section{Задание 1}

\subsection{Стационарное распределение вероятностей}

Марковский источник $X$ задан матрицей переходных вероятностей 
$$
\Pi=
\begin{pmatrix}
\frac{3}{4} & \frac{1}{4} & 0 \\ 
0 & \frac{1}{4} & \frac{3}{4} \\ 
\frac{1}{4} & \frac{1}{4} & \frac{1}{2} \end{pmatrix}$$ 
Полагаем, что процесс стационарный.
Значит, существует стационарнарное распределение $p$, задаваемое матрицей $\Pi.$
При этом $p$ является решением уравнения \begin{equation}\label{eq:peqpPi}p=p\Pi\end{equation}
Найдем стационарнарное распределение для заданного источника.
Для этого, используя уравнение \ref{eq:peqpPi}, составим систему:
$$
\begin{cases}
\left( \begin{matrix} a & b & c \end{matrix} \right) =
\begin{pmatrix}a & b & c \end{pmatrix}
\begin{pmatrix}
\frac{3}{4} & \frac{1}{4} & 0 \\
0 & \frac{1}{4} & \frac{3}{4} \\
\frac{1}{4} & \frac{1}{4} & \frac{1}{2}
\end{pmatrix} \\
a + b + c = 1
\end{cases}
$$
Отсюда получаем:
$$
\begin{cases} 
a = \frac{3}{4}a + \frac{1}{4}c \\
b = \frac{1}{4}a + \frac{1}{4}b + \frac{1}{4}c\\
a + b + c = 1
\end{cases}
$$
Откуда находим стационарнарное распределение $p = \left( \frac{3}{8} ,\frac{1}{4} ,\frac{3}{8} \right).$

\subsection{Энтропия источника}

\subsubsection{$\mathbf{H(X)}$}
Энтропия источника вычисляется по формуле $$H(X) = -\sum\limits_{i = 1} ^{3} p_i\log{p_i}$$ и равна: 
$H(X) = -\frac{3}{8}\log{\frac{3}{8}} - \frac{1}{4}\log{\frac{1}{4}} - \frac{3}{8}\log{\frac{3}{8}} = \frac{1}{4} \cdot \left( 11 - 3\log{3}\right) = 1.561.$

\subsubsection{$\mathbf{H(X|X^{\infty})}$}
Полагаем, что заданный марковский источник связности 1.
Тогда: $$H(X|X^{\infty}) = H(X|X)$$
По определению условной энтропии: $$H(X|X)= \sum\limits_{i=1}^{3}{p_{i}{H(X|p_{i})}} = -\sum\limits_{i=1}^{3}{p_{i}\sum\limits_{j=1}^{3}{\pi_{ij}\log{\pi_{ij}}}}$$
То есть $H(X|X^{\infty}) = -(\frac{3}{8}(\frac{3}{4}\log{\frac{3}{4}} + \frac{1}{4}\log{\frac{1}{4}}) +
\frac{1}{4}(\frac{1}{4}\log{\frac{1}{4}} + \frac{3}{4}\log{\frac{3}{4}}) +
\frac{3}{8}(\frac{1}{4}\log{\frac{1}{4}} + \frac{1}{4}\log{\frac{1}{4}} + \frac{1}{2}\log{\frac{1}{2}})) = \frac{29}{16} - \frac{15}{32}\log{3} = 1.07$.

\subsubsection{$\mathbf{H_{2}(X)}$ и $\mathbf{H_{n}(X)}$}
По определению: $$H_{n}(X) = \frac{H(X^n)}{n}$$
По свойству 3 условной энтропии: $$H(X^n) = H(X) + H(X|X) + \ldots +H(X|X^{n - 1})$$
Учитывая марковость и стационарность источника, получаем:
$$H_{n}(X) = \frac{H(X) + (n - 1)H(X|X)}{n} = H(X|X) + \frac{1}{n}(H(X) - H(X|X))$$
Значит $H_{n}(X) = (\frac{29}{16} - \frac{15}{32}\log{3}) + \frac{1}{n}(\frac{15}{16} - \frac{9}{32}\log{3}) = 1.07 + \frac{1}{n}0.492.$

Исходя из общей формулы, находим, что $H_{2}(X) = 1.07 + \frac{1}{2} \cdot 0.492 = 1.316.$

\subsection{Коды Хаффмена}

\subsubsection{Код Хаффмена для $\mathbf{X}$}
Построим код Хаффмена для $X$:

$$
\begin{tikzpicture}[frontier/.style={distance from root=250}, grow'=left]
\tikzset{level distance=1.2in,sibling distance=.1in}
\tikzset{edge from parent/.style= 
            {draw},
         every tree node/.style=
            {draw,minimum width=0.2in,align=center}}
\Tree[.1
         [.$\frac{5}{8}$
             [{{$c$} {$\frac{3}{8}$}\\код {$11$}} ]
             [{{$b$} {$\frac{1}{4}$}\\код {$10$}} ] ]
         [{{$a$} {$\frac{3}{8}$}\\код {$0$}} ] ]
\end{tikzpicture}
$$

Посчитаем скорость кодирования: $\overline{R} = \overline{l} = 1 \cdot \frac{3}{8} + 2 \cdot \frac{1}{4} + 2 \cdot \frac{3}{8} = \frac{13}{8} = 1.625$

Посчитаем избыточность: $ r = \overline{R} - H(X|X^{\infty}) = 1.625 - 1.07 = 0.555$ 

\subsubsection{Код Хаффмена для $\mathbf{X^2}$}
Построим код Хаффмена для $X^2$:
$$
\begin{tikzpicture}[frontier/.style={distance from root=350}, grow'=left]
\tikzset{level distance=1in,sibling distance=.1in}
\tikzset{edge from parent/.style= 
            {draw},
         every tree node/.style=
            {draw,minimum width=0.2in,align=center}}
\Tree[.1
        [.$\frac{3}{8}$
            [{{$cc$} {$\frac{3}{16}$}\\код {$11$}} ]
            [.$\frac{3}{16}$
                [{{$cb$} {$\frac{3}{32}$}\\код {$101$}} ]
                [{{$ca$} {$\frac{3}{32}$}\\код {$100$}} ] ] ]
        [.$\frac{5}{8}$
            [.$\frac{11}{32}$
                [{{$bc$} {$\frac{3}{16}$}\\код {$011$}} ]
                [.$\frac{5}{32}$
                    [{{$bb$} {$\frac{1}{16}$}\\код {$0101$}} ]
                    [{{$ab$} {$\frac{3}{32}$}\\код {$0100$}} ] ] ]
            [{{$aa$} {$\frac{9}{32}$}\\код {$00$}} ] ] ]
\end{tikzpicture}
$$

Посчитаем скорость кодирования: $\overline{R} =\frac{1}{2}\overline{l} =
\frac{1}{2}(2 \cdot \frac{9}{32} + 4 \cdot \frac{3}{32} + 4 \cdot \frac{1}{16} + 3 \cdot \frac{3}{16} + 3 \cdot \frac{3}{32} + 3 \cdot \frac{3}{32} + 2 \cdot \frac{3}{16}) = \frac{43}{32} = 1.344$

Посчитаем избыточность: $ r = \overline{R} - H(X|X^{\infty}) = 1.344 - 1.07 = 0.274$

\subsection{Наилучшее кодирование}

Заметим, что кодирование по Маркову является оптимальным для данного источника. 


\begin{tabular}{|c|c|c|c|}
\hline
    & A & B & C\\
\hline
a & 0 & 1 & {---}\\
\hline
b & {---} & 0 & 1\\
\hline
c & 00 & 01 & 1\\
\hline
средняя длина & $\frac{3}{2}$ & $\frac{4}{3}$ & 1\\
\hline
\end{tabular}

Посчитаем скорость кодирования: $\overline{R} = \frac{3}{2} \cdot \frac{3}{8} + \frac{4}{3} \cdot \frac{1}{4} + 1 \cdot \frac{3}{8} = 1,27$

Посчитаем избыточность: $ r = \overline{R} - H(X|X^{\infty}) = 1.27 - 1.07 = 0.2$ 
