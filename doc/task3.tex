\section{Задание 3}
Дан текст: "Четыре$\_$чертенка$\_$чертили$\_$черными$\_$чернилами$\_$чертеж." Требуется применить к нему различные методы сжатия.
Построим таблицу частот встречаемости символов в сообщении.

\begin{tabular}{|c|c|c|}
\hline
Символ & Число появлений & Вероятность появления\\
\hline
. & 1 & $\frac{1}{49}$\\
\hline
Ч & 1 & $\frac{1}{49}$\\
\hline
к & 1 & $\frac{1}{49}$\\
\hline
ж & 1 & $\frac{1}{49}$\\
\hline
ы & 2 & $\frac{2}{49}$\\
\hline
л & 2 & $\frac{2}{49}$\\
\hline
м & 2 & $\frac{2}{49}$\\
\hline
а & 2 & $\frac{2}{49}$\\
\hline
н & 3 & $\frac{3}{49}$\\
\hline
т & 4 & $\frac{4}{49}$\\
\hline
$\_$ & 5 & $\frac{5}{49}$\\
\hline
ч & 5 & $\frac{5}{49}$\\
\hline
и & 5 & $\frac{5}{49}$\\
\hline
р & 6 & $\frac{6}{49}$\\
\hline
е & 9 & $\frac{9}{49}$\\
\hline
\end{tabular}

\subsection{Алгоритм Хаффмена}

$$
\begin{tikzpicture}[frontier/.style={distance from root=450}, grow'=left]
\tikzset{level distance=0.7in,sibling distance=.09in}
\tikzset{edge from parent/.style= 
            {draw},
         every tree node/.style=
            {draw,minimum width=0.2in,align=center}}

\Tree[.1
        [.$\frac{19}{49}$
            [{{е} {$\frac{9}{49}$}\\код {$00$}} ]
            [.$\frac{10}{49}$
                [{{и} {$\frac{5}{49}$}\\код {$010$}} ]
                [{{ч} {$\frac{5}{49}$}\\код {$011$}} ] ] ]
        [.$\frac{30}{49}$
            [.$\frac{13}{49}$
                [{{р} {$\frac{6}{49}$}\\код {$100$}} ]
                [.$\frac{7}{49}$
                    [{{н} {$\frac{3}{49}$}\\код {$1010$}} ]
                    [.$\frac{4}{49}$
                        [{{а} {$\frac{2}{49}$}\\код {$10110$}} ]
                        [{{м} {$\frac{2}{49}$}\\код {$10111$}} ] ] ] ]
            [.$\frac{17}{49}$
                [.$\frac{9}{49}$
                    [{{$\_$} {$\frac{5}{49}$}\\код {$1100$}} ]
                    [{{т} {$\frac{4}{49}$}\\код {$1101$}} ] ]
                [.$\frac{8}{49}$
                    [.$\frac{4}{49}$
                        [{{л} {$\frac{2}{49}$}\\код {$11100$}} ]
                        [{{ы} {$\frac{2}{49}$}\\код {$11101$}} ] ]
                    [.$\frac{4}{49}$
                        [.$\frac{2}{49}$
                            [{{ж} {$\frac{1}{49}$}\\код {$111100$}} ]
                            [{{к} {$\frac{1}{49}$}\\код {$111101$}} ] ]
                        [.$\frac{2}{49}$
                            [{{Ч} {$\frac{1}{49}$}\\код {$111110$}} ]
                            [{{.} {$\frac{1}{49}$}\\код {$111111$}} ] ] ] ] ] ] ]
\end{tikzpicture}
$$

Закодируем исходный текст. Затраты на сообщение равны $l_2 = 178$ бит. Оценим затраты на передачу дерева.

\begin{tabular}{|c|c|c|}
\hline
Ярус & Число вершин & Число листьев\\
\hline
1 & 1 & 0\\
\hline
2 & 2 & 0\\
\hline
3 & 4 & 1\\
\hline
4 & 6 & 3\\
\hline
5 & 6 & 3\\
\hline
6 & 6 & 4\\
\hline
7 & 4 & 4\\
\hline
\end{tabular}

На передачу дерева потребуется $l_1 = 29 + 15 \cdot 8 = 149$ бит.Таким образом, суммарно потребуется $l = l_1 + l_2 = 327$ бит.

\subsection{Адаптивное A-кодирование}

\begin{tabular}{|c|c|c|}
\hline
Шаг & Символ & Вероятность появления\\
\hline
1 & Ч & $\frac{1}{1} \cdot \frac{1}{256}$\\
\hline
2 & е & $\frac{1}{2} \cdot \frac{1}{255}$\\
\hline
3 & т & $\frac{1}{3} \cdot \frac{1}{254}$\\
\hline
4 & ы & $\frac{1}{4} \cdot \frac{1}{253}$\\
\hline
5 & р & $\frac{1}{5} \cdot \frac{1}{252}$\\
\hline
6 & е & $\frac{1}{6}$\\
\hline
7 & $\_$ & $\frac{1}{7} \cdot \frac{1}{251}$\\
\hline
8 & ч & $\frac{1}{8} \cdot \frac{1}{250}$\\
\hline
9 & е & $\frac{2}{9}$\\
\hline
10 & р & $\frac{1}{10}$\\
\hline
11 & т & $\frac{1}{11}$\\
\hline
12 & е & $\frac{3}{12}$\\
\hline
13 & н & $\frac{1}{13} \cdot \frac{1}{249}$\\
\hline
14 & к & $\frac{1}{14} \cdot \frac{1}{248}$\\
\hline
15 & а & $\frac{1}{15} \cdot \frac{1}{247}$\\
\hline
16 & $\_$ & $\frac{1}{16}$\\
\hline
17 & ч & $\frac{1}{17}$\\
\hline
18 & е & $\frac{4}{18}$\\
\hline
19 & р & $\frac{2}{19}$\\
\hline
20 & т & $\frac{2}{20}$\\
\hline
21 & и & $\frac{1}{21} \cdot \frac{1}{246}$\\
\hline
22 & л & $\frac{1}{22} \cdot \frac{1}{245}$\\
\hline
23 & и & $\frac{1}{23}$\\
\hline
24 & $\_$ & $\frac{2}{24}$\\
\hline
25 & ч & $\frac{2}{25}$\\
\hline
26 & е & $\frac{5}{26}$\\
\hline
27 & р & $\frac{3}{27}$\\
\hline
28 & н & $\frac{1}{28}$\\
\hline
29 & ы & $\frac{1}{29}$\\
\hline
30 & м & $\frac{1}{30} \cdot \frac{1}{244}$\\
\hline
31 & и & $\frac{2}{31}$\\
\hline
32 & $\_$ & $\frac{3}{32}$\\
\hline
33 & ч & $\frac{3}{33}$\\
\hline
34 & е & $\frac{6}{34}$\\
\hline
35 & р & $\frac{4}{35}$\\
\hline
36 & н & $\frac{2}{36}$\\
\hline
37 & и & $\frac{3}{37}$\\
\hline
38 & л & $\frac{1}{38}$\\
\hline
39 & а & $\frac{1}{39}$\\
\hline
40 & м & $\frac{1}{40}$\\
\hline
41 & и & $\frac{4}{41}$\\
\hline
42 & $\_$ & $\frac{4}{42}$\\
\hline
43 & ч & $\frac{4}{43}$\\
\hline
44 & е & $\frac{7}{44}$\\
\hline
45 & р & $\frac{5}{45}$\\
\hline
46 & т & $\frac{3}{46}$\\
\hline
47 & е & $\frac{8}{47}$\\
\hline
48 & ж & $\frac{1}{48} \cdot \frac{1}{243}$\\
\hline
49 & . & $\frac{1}{49} \cdot \frac{1}{242}$\\
\hline
\end{tabular}

Затраты на сообщение будут равны $\lceil-\log(G)\rceil + 1 = 290$ бит.

\subsection{Нумерационное кодирование}

Найдем затраты на передачу первой части кода. Для этого упорядочим частоты появления символов по невозрастанию.

$\tau = (9, 6, 5, 5, 5, 4, 3, 2, 2, 2, 2, 1, 1, 1, 1, 0, \ldots, 0)$

Затраты на передачу такой композиции при помощи арифметического кодирования составляют
$\lceil\log(n \cdot \prod\limits_{j;\tau_j>0}\tau_j)\rceil = 26$ бит.

Посчитаем частоту появления каждого числа в $\tau$: $\tau' = (1, 1, 3, 1, 1, 4, 4, 241)$.

Затраты на передачу соответствия между буквами алфавита и компонентами $\tau$ вычисляются
по следующей формуле:

$\lceil\log(\frac{256!}{\prod\limits_{j}\tau_j'!})\rceil = 108$ бит.

Итого затраты на передачу первой части нумерационного кода составляют $134$ бита.

Затраты на передачу второй части нумерационного кода вычисляются по следующей формуле:
$\lceil\log(\frac{n!}{\prod\nolimits_{j=0}^{255}\tau_j!})\rceil = 149$ бит.

Общие затраты на передачу сообщения составляют $283$ бита.

\subsection{LZ-77}

В качестве монотонного кодирования используется унарное кодирование.

\begin{tabular}{|c|c|c|c|c|c|c|}
\hline
Шаг & Флаг & Фрагмент & d & l & Кодовое слово & Длина\\
\hline
1 & 0 & Ч &  & 0 & 0 bin(Ч) & 9\\
\hline
2 & 0 & е &  & 0 & 0 bin(е) & 9\\
\hline
3 & 0 & т &  & 0 & 0 bin(т) & 9\\
\hline
4 & 0 & ы &  & 0 & 0 bin(ы) & 9\\
\hline
5 & 0 & р &  & 0 & 0 bin(р) & 9\\
\hline
6 & 1 & е & 3 & 1 & 1 011 0 & 5\\
\hline
7 & 0 & $\_$ &  & 0 & 0 bin($\_$) & 9\\
\hline
8 & 0 & ч &  & 0 & 0 bin(ч) & 9\\
\hline
9 & 1 & е & 6 & 1 & 1 110 0 & 5\\
\hline
10 & 1  & р & 4 & 1 & 1 0100 0 & 6\\
\hline
11 & 1 & т & 7 & 1 & 1 0111 0 & 6\\
\hline
12 & 1 & е & 9 & 1 & 1 1001 0 & 6\\
\hline
13 & 0 & н &  & 0 & 0 bin(н) & 9\\
\hline
14 & 0 & к &  & 0 & 0 bin(к) & 9\\
\hline
15 & 0 & а &  & 0 & 0 bin(а) & 9\\
\hline
16 & 1 & $\_$черт & 8 & 5 & 1 1000 11110 & 10\\
\hline
17 & 0 & и &  & 0 & 0 bin(и) & 9\\
\hline
18 & 0 & л &  & 0 & 0 bin(л) & 9\\
\hline
19 & 1 & и & 1 & 1 & 1 00001 0 & 7\\
\hline
20 & 1 & $\_$чер & 16 & 4 & 1 10000 1110 & 10\\
\hline
21 & 1 & н & 14 & 1 & 1 01110 0 & 7\\
\hline
22 & 1 & ы & 24 & 1 & 1 10110 0 & 7\\
\hline
23 & 0 & м &  & 0 & 0 bin(м) & 9\\
\hline
24 & 1 & и$\_$черн & 7 & 6 & 1 00111 111110 & 12\\
\hline
25 & 1 & ил & 15 & 2 & 1 001111 10 & 9\\
\hline
26 & 1 & а & 23 & 1 & 1 010111 0 & 8\\
\hline
27 & 1 & ми$\_$чер & 9 & 6 & 1 001001 111110 & 13\\
\hline
28 & 1 & те & 34 & 2 & 1 100010 10 & 9\\
\hline
29 & 0 & ж &  & 0 & 0 bin(ж) & 9\\
\hline
30 & 0 & . &  & 0 & 0 bin(.) & 9\\
\hline
\end{tabular}

Длина закодированного сообщения составляет 255 бит.

\subsection{LZ-78}

\begin{tabular}{|c|c|c|c|c|}
\hline
Шаг & Словарь & Номер слова & Кодовое слово & Затраты\\
\hline
0 & esc & {---} &  & \\
\hline
1 & Ч & 0 & bin(Ч) & 8\\
\hline
2 & е & 0 & bin(е) & 8\\
\hline
3 & т & 0 & 0 bin(т) & 9\\
\hline
4 & ы & 0 & 00 bin(ы) & 10\\
\hline
5 & р & 0 & 00 bin(р) & 10\\
\hline
6 & е$\_$ & 2 & 110 & 3\\
\hline
7 & $\_$ & 0 & 000 bin($\_$) & 11\\
\hline
8 & ч & 0 & 000 bin(ч) & 11\\
\hline
9 & ер & 2 & 010 & 3\\
\hline
10 & рт & 5 & 0101 & 4\\
\hline
11 & те & 3 & 0011 & 4\\
\hline
12 & ен & 2 & 0010 & 4\\
\hline
13 & н & 0 & 0000 bin(н) & 12\\
\hline
14 & к & 0 & 0000 bin(к) & 12\\
\hline
15 & а & 0 & 0000 bin(а) & 12\\
\hline
16 & $\_$ч & 7 & 0111 & 4\\
\hline
17 & че & 8 & 1000 & 4\\
\hline
18 & ерт & 9 & 01001 & 5\\
\hline
19 & ти & 3 & 00011 & 5\\
\hline
20 & и & 0 & 00000 bin(и) & 13\\
\hline
21 & л & 0 & 00000 bin(л) & 13\\
\hline
22 & и$\_$ & 20 & 10100 & 5\\
\hline
23 & $\_$че & 16 & 10000 & 5\\
\hline
24 & ерн & 9 & 01001 & 5\\
\hline
25 & ны & 13 & 01101 & 5\\
\hline
26 & ым & 4 & 00100 & 5\\
\hline
27 & м & 0 & 00000 bin(м) & 13\\
\hline
28 & и$\_$ч & 22 & 10110 & 5\\
\hline
29 & чер & 17 & 10001 & 5\\
\hline
30 & рн & 5 & 00101 & 5\\
\hline
31 & ни & 13 & 01101 & 5\\
\hline
32 & ил & 20 & 10100 & 5\\
\hline
33 & ла & 21 & 10101 & 5\\
\hline
34 & ам & 15 & 001111 & 6\\
\hline
35 & ми & 27 & 011011 & 6\\
\hline
36 & и$\_$че & 28 & 011100 & 6\\
\hline
37 & ерте & 18 & 010010 & 6\\
\hline
38 & еж & 2 & 000010 & 6\\
\hline
39 & ж & 0 & 000000 bin(ж) & 14\\
\hline
40 & . & 0 & 000000 bin(ю) & 14\\
\hline
\end{tabular}

Длина закодированного сообщения составляет 291 бит.

\subsection{PPMA}

\begin{tabular}{|c|c|c|c|c|c|}
\hline
Шаг & Символ & Контекст s & $\tau_t(s)$ & $\hat p_t(\varepsilon|s)$ & $\hat p_t(a|s)$\\
\hline
1 & Ч & $\#$ & 0 & $\frac{1}{1}$ & $\frac{1}{256}$\\
\hline
2 & е & $\#$ & 1 & $\frac{1}{2}$ & $\frac{1}{255}$\\
\hline
3 & т & $\#$ & 2 & $\frac{1}{3}$ & $\frac{1}{254}$\\
\hline
4 & ы & $\#$ & 3 & $\frac{1}{4}$ & $\frac{1}{253}$\\
\hline
5 & р & $\#$ & 4 & $\frac{1}{5}$ & $\frac{1}{252}$\\
\hline
6 & е & $\#$ & 5 &  & $\frac{1}{6}$\\
\hline
7 & $\_$ & е & 1, 5 & $\frac{1}{2} \cdot \frac{1}{6}$ & $\frac{1}{251}$\\
\hline
8 & ч & $\#$ & 7 & $\frac{1}{8}$ & $\frac{1}{250}$\\
\hline
9 & е & $\#$ & 8 &  & $\frac{2}{9}$\\
\hline
10 & р & е & 2, 7 & $\frac{1}{3}$ & $\frac{1}{8}$\\
\hline
11 & т & р & 1, 7 & $\frac{1}{2}$ & $\frac{1}{8}$\\
\hline
12 & е & т & 1, 10 & $\frac{1}{2}$ & $\frac{3}{11}$\\
\hline
13 & н & е & 3, 7 & $\frac{1}{4} \cdot \frac{1}{8}$ & $\frac{1}{249}$\\
\hline
14 & к & $\#$ & 13 & $\frac{1}{14}$ & $\frac{1}{248}$\\
\hline
15 & а & $\#$ & 14 & $\frac{1}{15}$ & $\frac{1}{247}$\\
\hline
16 & $\_$ & $\#$ & 15 &  & $\frac{1}{16}$\\
\hline
17 & ч & $\_$ & 1 &  & $\frac{1}{2}$\\
\hline
18 & е & $\_$ч & 1 &  & $\frac{1}{2}$\\
\hline
19 & р & $\_$че & 1 &  & $\frac{1}{2}$\\
\hline
20 & т & $\_$чер & 1 &  & $\frac{1}{2}$\\
\hline
21 & и & $\_$черт & 1, 0, 0, 0, 1, 14 & $\frac{1}{2} \cdot \frac{1}{1} \cdot \frac{1}{1} \cdot \frac{1}{1} \cdot \frac{1}{2} \cdot \frac{1}{15}$ & $\frac{1}{246}$\\
\hline
22 & л & $\#$ & 21 & $\frac{1}{22}$ & $\frac{1}{245}$\\
\hline
23 & и & $\#$ & 22 &  & $\frac{1}{23}$\\
\hline
24 & $\_$ & и & 1, 22 & $\frac{1}{2}$ & $\frac{2}{23}$\\
\hline
25 & ч & $\_$ & 2 &  & $\frac{2}{3}$\\
\hline
26 & е & $\_$ч & 2 &  & $\frac{2}{3}$\\
\hline
27 & р & $\_$че & 2 &  & $\frac{2}{3}$\\
\hline
28 & н & $\_$чер & 2, 0, 0, 1, 18 & $\frac{1}{3} \cdot \frac{1}{1} \cdot \frac{1}{1} \cdot \frac{1}{2}$ & $\frac{1}{19}$\\
\hline
29 & ы & н & 1, 27 & $\frac{1}{2}$ & $\frac{1}{28}$\\
\hline
30 & м & ы & 1, 25 & $\frac{1}{2} \cdot \frac{1}{26}$ & $\frac{1}{244}$\\
\hline
31 & и & $\#$ & 30 &  & $\frac{2}{31}$\\
\hline
32 & $\_$ & и & 2 &  & $\frac{1}{3}$\\
\hline
33 & ч & и$\_$ & 1 &  & $\frac{1}{2}$\\
\hline
34 & е & и$\_$ч & 1 &  & $\frac{1}{2}$\\
\hline
35 & р & и$\_$че & 1 &  & $\frac{1}{2}$\\
\hline
36 & н & и$\_$чер & 1 &  & $\frac{1}{2}$\\
\hline
37 & и & $\_$черн & 1, 0, 0, 0, 1, 33 & $\frac{1}{2} \cdot \frac{1}{1} \cdot \frac{1}{1} \cdot \frac{1}{1} \cdot \frac{1}{2}$ & $\frac{3}{34}$\\
\hline
38 & л & и & 3 &  & $\frac{1}{4}$\\
\hline
39 & а & ил & 1, 0, 34 & $\frac{1}{2} \cdot \frac{1}{1}$ & $\frac{1}{35}$\\
\hline
40 & м & а & 1, 35 & $\frac{1}{2}$ & $\frac{1}{36}$\\
\hline
41 & и & м & 1 &  & $\frac{1}{2}$\\
\hline
42 & $\_$ & ми & 1 &  & $\frac{1}{2}$\\
\hline
43 & ч & ми$\_$ & 1 &  & $\frac{1}{2}$\\
\hline
44 & е & ми$\_$ч & 1 &  & $\frac{1}{2}$\\
\hline
45 & р & ми$\_$че & 1 &  & $\frac{1}{2}$\\
\hline
46 & т & и$\_$чер & 2, 2 & $\frac{1}{3}$ & $\frac{2}{3}$\\
\hline
47 & е & $\_$черт & 2 &  & $\frac{1}{3}$\\
\hline
48 & ж & черте & 1, 0, 0, 0, 7, 29 & $\frac{1}{2} \cdot \frac{1}{1} \cdot \frac{1}{1} \cdot \frac{1}{1} \cdot \frac{1}{8} \cdot \frac{1}{30}$ & $\frac{1}{243}$\\
\hline
49 & . & $\#$ & 48 & $\frac{1}{49}$ & $\frac{1}{242}$\\
\hline
\end{tabular}

Затраты на сообщение составляют 263 бита.

\subsection{Результаты}

\begin{tabular}{|c|c|c|}
\hline
Алгоритм & Размер сжатого сообщения & Коэффициент сжатия\\
\hline
Алгоритм Хаффмена & 327 & 1.199\\
\hline
Адаптивное A-кодирование & 290 & 1.352\\
\hline
Нумерационное кодирование & 283 & 1.385\\
\hline
\bf{LZ-77} & \bf{255} & \bf{1.537}\\
\hline
LZ-78 & 291 & 1.347\\
\hline
PPMA & 263 & 1.490\\
\hline
gzip & 320 & 1.225\\
\hline
\end{tabular}
